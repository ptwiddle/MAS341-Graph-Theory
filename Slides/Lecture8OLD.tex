\documentclass{beamer}
\beamertemplatenavigationsymbolsempty
\usepackage{amsmath, amssymb, hyperref, graphics}
\usepackage{tikz}



\title{Graph Theory Lecture 8}

\begin{document}


\begin{frame}{Spanning trees}
Trees are the minimal connected graphs.  Spanning trees are minimal subgraphs that contain all the vertices but are connected.
  \begin{definition}Let $G$ be a connected graph.  A \emph{spanning tree} of $G$ is a subgraph $T\subseteq G$ such that $T$ is a tree, and $T$ contains every vertex of $G$.
    \end{definition}
  \begin{block}{Side point: Kirkchoff's Matrix Tree Theorem}
    Spanning trees of $K_n$ are the same thing as labelled trees on $n$ vertices. \\
As a generalization of Cayley's formula, can compute the number of spanning trees of any graph $G$ as the determinant of a matrix.
    \end{block}

\end{frame}

\begin{frame}{Weighted graphs}
Edges often have a ``cost'' associated to them -- the time, money, or distance of the corresponding route/connection.

  \begin{definition}{Weighted graph}
    A \emph{weighted graph} is a graph $G$ together with a weight function $w:E(G)\to \mathbb{R}.$  Normally we assume $w(e)\geq 0$ for all edges $e$.
  \end{definition}

  Weighted graphs are often encoded in tables: 
\begin{center}
  \begin{tabular}{c|c|c|c|c}
    A &   &   &   &   \\ \hline
    4 & B &   &   &   \\ \hline
    5 & 7 & C &   &   \\ \hline
    9 & 8 & 8 & D &   \\ \hline
    5 & 5 & 5 & 8 & E \\
    \end{tabular}
  \end{center}
  \end{frame}

\begin{frame}{Minimal spanning trees}
\begin{block}{Motivating problem:}
  Suppose that the vertices of a weighted graph $G$ represented cities, and the weight $w(e)$ of an edge was the cost of building a road between the cities.  What's the cheapest way to connect all the cities?
\end{block}
  \begin{definition} Let $T\subseteq G$ be a spanning tree of a weighted graph.  The weight of $T$ is the total weight of all its edges:
    $$w(T)=\sum_{e\in T}w(e)$$
 \end{definition}
  \begin{block}{Problem becomes: find the minimal weight spanning tree}
    Checking every spanning tree too slow: $K_n$ has $n^{n-2}$
    \end{block}
\end{frame}

\begin{frame}{Many solutions.  Two: Kruskal and Prim}
  \begin{block}{Loose concept: Greedy Algorithms}
    A \emph{greedy algorithm} doesn't plan ahead, but just does the best it can at each stage.
  \end{block}

  \begin{definition}[Kruskal's algorithm]
Start with $T$ having no edges. Iteratively:
    \begin{itemize}
    \item Look at cheapest remaining edge $e$
    \item If adding $e$ to $T$ creates a loop, discard $e$
    \item Otherwise, add $e$ to $T$
      \end{itemize}
\end{definition}
  \begin{block}{Fairly clear: produces a spanning tree}
    But it's \emph{not} clear this spanning tree is \emph{minimal}.
  \end{block}
\end{frame}

  
  \begin{frame}{Another approach: Prim}
\begin{block}{Kruskal:  a global view, ``avoid cycles''}
  \begin{itemize}
  \item Kruskal's algorithm looks at all edges at start
  \item $T$ may be disconnected at intermediate steps
    \end{itemize}
\end{block}
\begin{block}{Prim: local view, ``build tree''}
  Start at one vertex and explore out
\end{block}

\begin{definition}[Prim's algorithm]
  Start $T={v}$, a single vertex.  Iteratively:
  \begin{itemize}    \item Find the cheapest edge $e=vw$ from $v\in T$ to $w\notin T$
  \item Add $e$ and $w$ to $T$
  \end{itemize}
\end{definition}
\begin{block}{Fairly clear: produces a spanning tree}
  But it's \emph{not} clear this spanning tree is \emph{minimal}.
  \end{block}

    

\end{frame}
  \begin{frame}{Why do Kruskal and Prim work?}
    \begin{block}{Exchange principle:}
      Let $T$ be a spanning tree of $G$, and $e=xy$ an edge not in $T$. Then:
      \begin{itemize}
      \item Unique path $P$ from $x$ to $y$ using only edges of $T$
      \item If $f$ any edge in $P$, then $T^\prime=T\setminus f\cup e$ a spanning tree
      \end{itemize}
i.e., can exchange edges in $P$ for $e$.
    \end{block}

    \begin{block}{Basic idea of proofs:}
      \begin{itemize}
        \item Let $T$ be spanning tree produced by algorithm 
\item  Let $T_m$ be a minimal spanning tree 
     \item  Transform $T_m$ to $T$ edge by edge using exchange principle 
\item   Show each step is a minimal spanning tree
\end{itemize}
Key: always add cheapest edge in $T$ but not $T_m$.
    \end{block}
\end{frame}
  \begin{frame}{Finding \emph{all} minimal spanning trees}
    \begin{block}{All edges have distinct weights:}
      \begin{itemize}
        \item Never have to make an arbitrary choice
        \item      Unique minimal weight spanning tree
          \end{itemize}
    \end{block}
    \begin{block}{All edges have the same weight:}
      \begin{itemize}
        \item Any spanning tree is minimal 
        \item      Probably too many to write down
          \end{itemize}
    \end{block}
    \begin{block}{A few edges have repeated weights:}
      \begin{itemize} \item Only a few places we have to make a choice
      \item Can find them with case by case analysis breaking ties
      \end{itemize}
    \end{block}
    \end{frame}
  
\end{document}
