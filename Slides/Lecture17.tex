\documentclass{beamer}
\beamertemplatenavigationsymbolsempty
\usepackage{graphicx, tikz}


\begin{document}


\begin{frame}{Chromatic Number}
  \begin{theorem} For $G$ a planar graph $\chi(G)\leq 6$.
\end{theorem}

  \begin{block}{Step 1: Using Euler's Theorem}
    \begin{definition}$\delta(G)$ denotes the \emph{minimum} degree of all vertices in $G$.
      \end{definition}
    \begin{lemma}For $G$ a simple planar graph $\delta(G)\leq 5$.
    \end{lemma}  
        \end{block}
  
  \begin{block}{Step 2: Induction}
 We proved $\chi(G)\leq \Delta(G)+1$ by colouring the vertices of $G$ in any order.  The Lemma bounds $\delta(G)$ and not $\Delta(G)$; need to be a little smarter. 
\end{block}
    
\end{frame}

\begin{frame}{Proof that simple planar graphs have $\delta(G)\leq 5$}
  Assume not, then every vertex has $d(v)\geq 6$.
\begin{block}{The three ingredients:}
  \begin{itemize}
  \item Euler's Theorem $V-E+F=2$
  \item Face-Edge handshaking \\
    Simple, so $d(f)\geq 3$ for all faces. So $2E\geq 3F$
  \item Vertex-Edge handshaking \\
    By assumption, $d(v)\geq 6$, so $2E\geq 6V$
    \end{itemize}


\end{block}
 
\end{frame}

\begin{frame}{Proof of the Six Colour Theorem}
  Assume $G$ is planar.  We can assume that $G$ is simple.  Why? 

  \begin{block}{Induct on $n$ the number of vertices}
  \end{block}


  \begin{block}{Base case: $n\leq 6$}
    At most six vertices, so can give each vertex a different colour.
  \end{block}

\begin{block}{Inductive Step}
Assume that $G$ has $n$ vertices, and every planar simple graph with less than $n$ vertices can be coloured with six colours. \\
\begin{itemize}
\item By the Lemma, $G$ has a vertex $v$ with $d(v)\leq 5$
\item The graph $G\setminus v$ has $n-1$ vertices, so can be six coloured
\item Now colour $v$
\end{itemize}
\end{block}
 \end{frame}

\begin{frame}{Chromatic index}Suppose six teams $A-F$ are in a soccer league, and each team will play three games:
  $$\begin{array}{cccccc}
A &   &   &   &   &   \\
X & B &   &   &   &   \\
X & X & C &   &   &   \\
X &   &   & D &   &   \\
  & X &   & X & E &   \\
   &   & X & X & X & F
    \end{array}$$
If each team plays one game a week, can the tournament be run in three weeks?  How about if we want $AB$ and $DE$ to play on different weeks?
\begin{block}{Make it a graph}
  \end{block}

\end{frame}

\begin{frame}{Chromatic index $\chi^\prime(G)$}
  \begin{definition}
The \emph{chromatic index} $\chi^\prime(G)$ denotes the minimum number of colours needed to colour the \emph{edges} of $G$ so that any two edges that share a vertex have different colours.
    \end{definition}
In the application: the colours were the weeks?

\begin{block}{Examples:}
  \begin{itemize}
  \item $\chi^\prime(K_4)=3$
    \item $\chi^\prime(K_5)=5$
  \end{itemize}
  \end{block}
\begin{lemma} $\chi^\prime(G)\geq \Delta(G)$
  \end{lemma}

\end{frame}

\begin{frame}{Finding $\chi^\prime(G)$}

  \begin{theorem}[Vizing]
For a simple graph $\chi^\prime(G)=\Delta$ or $\Delta+1$
    \end{theorem}
  We won't prove Vizing's Theorem, but will only implicitly use it.

  \begin{block}{One method to find $\chi^\prime(G)$ with proof:}
 We know $\chi^\prime(G)\geq \Delta(G)$.  Try to colour it with $\Delta(G)$.
 \begin{itemize}
 \item If we can, we have shown $\chi^\prime(G)=\Delta$
 \item If we can't, prove it: so we know $\chi^\prime(G)\geq \Delta+1$
 \item Find a colouring with $\Delta+1$ colours 
   \end{itemize}


   \end{block}
  \end{frame}

\begin{frame}{Another way to prove $\chi^\prime(G)\geq \Delta+1$}



  \begin{example}{$K_n$ for $n$ odd}
    Suppose $n=2k+1$ is odd.
\begin{itemize}
\item Then $\Delta(K_n)=n-1=2k$  
\item Note $K_n$ has $k(2k+1)$ edges.  Why? 
\item We can have at most $k$ edges of any given colour
\item So using $2k$ colours, can only colour $2k^2<k(2k+1)$ edges
\end{itemize}
  


    

\end{example}
\end{frame}

\end{document}
